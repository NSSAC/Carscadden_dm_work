\documentclass[11pt]{article}

\setlength{\textheight}{9.0 in}
\setlength{\textwidth}{6.5 in}
\setlength{\oddsidemargin}{0 in}
\setlength{\evensidemargin}{0 in}
\setlength{\topmargin}{-0.5 in}

\setlength{\parskip}{4pt}

%% \usepackage[nofillcomment,ruled,linesnumbered]{algorithm2e}
\usepackage{amsmath}
\usepackage{amssymb}
\usepackage{color}
\usepackage{graphicx}
\usepackage{url}

%% \pagestyle{empty}
\begin{document}

\newtheorem{theorem}{Theorem}[section]
\newtheorem{lemma}{Lemma}[section]
\newtheorem{corollary}{Corollary}[section]
\newtheorem{fact}{Fact}[section]
\newtheorem{definition}{Definition}[section]
\newtheorem{proposition}{Proposition}[section]
\newtheorem{observation}{Observation}[section]
\newtheorem{claim}{Claim}[section]

\newcommand{\cnp}{\textbf{NP}}
\newcommand{\true}{\texttt{True}}
\newcommand{\false}{\texttt{False}}

\newcommand{\QED}{\hfill\rule{2mm}{2mm}}

\newcommand{\irange}{\mbox{$1 \leq i \leq n$}}
\newcommand{\jrange}{\mbox{$1 \leq j \leq m$}}

\newcommand{\dunder}[1]{\underline{\underline{#1}}}

\setlength{\parskip}{3pt}

\baselineskip=1.1\normalbaselineskip

\begin{center}
\dunder{\Large{\textbf{Literature Review}}}
\end{center}

\bigskip

\noindent
\textbf{List of Papers Reviewed:}

\begin{enumerate}
\item   P. Kumar, P. Verma, A. Singh and H. Cherifi,
``Choosing Optimal Seed Nodes in Competitive Contagion”, 
\emph{Frontiers in Big Data}, Vol. 2, Article 6, June 2019, pp. 1--6.

\item S. Bharathi, D. Kempe and M. Salek, ``Competitive Influence 
Maximization in Social Networks”, \emph{Proc. Workshop on Internet Economics}
(WINE), 2007, pp. 306--311.

\item A. Adiga, C. J. Kuhlman, M. V. Marathe, H. S. Mortveit, 
S.~S. Ravi and A. Vullikanti, ``Graphical Dynamical Systems and Their
Applications to Bio-Social Systems", \emph{International Journal of Advances
in Engineering Sciences and Applied Mathematics}, Vol. 11, 2019, pp. 153--171.

\item M. E. J. Newman and C. R. Ferrario, ``Interacting Epidemics and 
Coinfection on Contact Networks", \emph{PLOS One}, Vol. 8, No. 8, Aug. 2013,
8 pages.

\item S. Myers and J. Leskovec, ``Clash of the Contagions: Cooperation and 
Competition in Information Diffusion", \emph{Proc. IEEE Intl. Conf. Data Mining}
(ICDM), 2012, pp. 539--548.

\item A. Beutel, B. A. Prakash, R. Rosenfeld and C. Faloutsos,
``Interacting Viruses in Networks: Can Both Survive?", 
\emph{Proc. ACM KDD}, 2012, pp. 426--434.

\item B. Karrer and M. E. J. Newman, ``Competing Epidemics on Complex Networks",
Arxiv Report arXiv:1105.3424v1, 2011, 14 pages.

\item Chen et al., ``Robust Influence Maximization", \emph{KDD '16: Proceedings of the 22nd ACM SIGKDD International Conference on Knowledge Discovery and Data Mining}, 2016, pgs. 795-804.

\item A. Stanoev, D. Trpevski,and L. Kocarev, ``Modeling the Spread of Multiple Concurrent Contagions on Networks", PlosOne Vol. 9 pp. 1--16.

\item Wei Chen, Chi Wang, and Yajun Wang. 2010. Scalable influence maximization for prevalent viral marketing in large-scale social networks. In Proceedings of the 16th ACM SIGKDD international conference on Knowledge discovery and data mining (KDD ’10). Association for Computing Machinery, New York, NY, USA, 1029–1038. 

\item Wei Chen, Yajun Wang, and Siyu Yang. 2009. Efficient influence maximization in social networks. In Proceedings of the 15th ACM SIGKDD international conference on Knowledge discovery and data mining (KDD ’09). Association for Computing Machinery, New York, NY, USA, 199–208. DOI:https://doi.org/10.1145/1557019.1557047  

\end{enumerate}

\clearpage

\noindent
\textbf{Paper:}~  P. Kumar, P. Verma, A. Singh and H. Cherifi, 
``Choosing Optimal Seed Nodes in Competitive Contagion”, 
\emph{Frontiers in Big Data}, Vol. 2, Article 6, June 2019, pp. 1--6.

\medskip

\noindent
\textbf{Summary:}

\begin{itemize}
\item Focused on influence maximization
  \begin{itemize}
     \item two competing infections
     \item seed nodes chosen according to centrality measure; 
           higher centrality nodes have a high costs; 
           cost of the nodes is determined by best centrality measure
    \item less quantity, more influential nodes v. more quantity, less influential
        \begin{itemize}
	   \item used best centrality measure for given dataset
	   \item high-less is best across the board for all datasets
        \end{itemize}
	\item studies appear from a game theoretic setting; use of Nash equilibrium
	\item undirected network
	\item measures: degree centrality, closeness centrality
	(1/sum of distance from other nodes), betweenness centrality
	(percentage of shortest paths v appears in), the maximum
	eigenvalue of a node’s eigenvalue in the adjacency matrix,
	pagerank
	\item choices made in first experiment by order of decreasing centrality
	\item no one centrality measure is best
	\item equal competitive constants
	\item no conclusive evidence on using one centrality measure
            \begin{itemize}
		\item search algorithm for picking nodes not in
		order of descending centrality but based off expected
		benefit
            \end{itemize}
	\item cascade algorithm attempts to spread to all adjacent
	nodes; every time it is infected the competitive constant
	for the infecting strain grows on that node
   \end{itemize}
\end{itemize}

\smallskip

\noindent
\rule{\textwidth}{0.01in}

\clearpage

\noindent
\textbf{Paper:}~
S. Bharathi, D. Kempe and M. Salek, ``Competitive Influence
Maximization in Social Networks”, \emph{Proc. Workshop on Internet Economics}
(WINE), 2007, pp. 306--311.

\medskip

\begin{enumerate}
    \item Model:
    \begin{enumerate}
        \item Diffusion model
        
        \item Each edge has an activation probability
        
        \item The first activated incoming edge of a node determines the which infection it has
        
        \item The time at which the edge activation is determined by continuous independent, exponential r.v.'s
        
        \item Weakness: Once a node is activated, it has no chance of changing sets.$\implies$ Every graph reaches a fixed point
    \end{enumerate}
    \item Agent $i$ wants to maximize $E[|T_{i}|]$
    \item Strategy for the last player to commit
    \begin{enumerate}
    
        \item Greedy choice of adding node with highest 
        marginal return is within $1-\frac{1}{e}$ of optimal
        
        \item Uses fact that the nodes chosen according to this strategy add at least as many paths as those according to another strategy
        
    \end{enumerate}
    
    \item The Nash Equilibrium expected set of activated nodes is at least half that of if one person played
    
    \item First mover strategy in duopoly
    \begin{enumerate}
        \item Strongly polynomial dynamic program to maximize r's pieces
    \end{enumerate}
    
    \item FPTAS available for tree version of problem
    
    \item No greedy algorithm for more than duopoly and not last player
    
    \item Terms: 
    \begin{enumerate}
    
    \item Submodular- as the number of neighbors activates increases, their marginal impact decreases
    \\ i.e. for $S,T \subset V$, $f(S) + f(T) \geq f(S \cap T) + f(S \cup T)$
    
    \item Monotonicity- as the number of activated neighbors grows, the likelihood of activation increases
    \\ i.e. $f(S) \leq f(T) \leq f(V)$ \\ for $S \subset T \subset V$
    
    \end{enumerate}
    
\end{enumerate}
\smallskip
\noindent
\rule{\textwidth}{0.01in}

\clearpage

\noindent
\textbf{Paper:}~
A. Adiga, C. J. Kuhlman, M. V. Marathe, H. S. Mortveit,
S.~S. Ravi and A. Vullikanti, ``Graphical Dynamical Systems and Their
Applications to Bio-Social Systems", \emph{International Journal of Advances
in Engineering Sciences and Applied Mathematics}, Vol. 11, 2019, pp. 153--171.


\begin{enumerate}
    \item Specified by $\mathcal{G, F, U}$
    \begin{enumerate}
        \item $\mathcal{F}$ uses inputs only from time $t$ to determine update at $t+1$
        \item $\mathcal{F}$ gives local transition functions
        \item a transition function $f$ is $r$ symmetric if it follows the threshold model i.e. $r=1,f(\Sigma a_{i}), r=2 f(\Sigma a_{i}, \Sigma a_{j}), ... etc.$ where if each of the sums is above a threshold $f$ is 1 and 0 otherwise.
    \end{enumerate}
    \item A Garden of Eden configuration has no predecessor
    \item A fixed-point is configuration that updates to itself
    
    \item Three analysis problems covered in depth:
    \begin{enumerate}
        \item Fixed-Point Existence
        \begin{enumerate}
            \item Checking is trivial
            \item NP-Complete in general but efficiently solvable in a lot of cases
        \end{enumerate}
        \item Reachability
        \begin{enumerate}
            \item Symmetric functions make this easier
            \item Intractable for heterogeneous symmetric functions
            \item Efficient for bi-threshold, acyclic, directed graphs
        \end{enumerate}
        \item Predecessor Existence
        \begin{enumerate}
            \item Existence + counting version are solvable for r-symmetric and treewidth bounded graphs
            \item In general, however, is NP-Complete
        \end{enumerate}
    \end{enumerate}
\end{enumerate}

\smallskip
\noindent
\rule{\textwidth}{0.01in}

\clearpage

\noindent
\textbf{Paper:}~ 
M. E. J. Newman and C. R. Ferrario, ``Interacting Epidemics and
Coinfection on Contact Networks", \emph{PLOS One}, Vol. 8, No. 8, Aug. 2013,
8 pages.

\begin{enumerate}
    \item Model Assumptions:
    \begin{enumerate}
        \item Sequential Infections-Disease A first propagates through the network. Disease B then propagates after it.
        \item A node must first have been infected by disease A if it is to be infected by disease B
        \item Follows the SIR model
        \item Infected individuals become recovered at a fixed-time $\tau$ and infect neighbors with fixed probability $\beta$
        \item This model assumes the same mode of transmission (This is required for the network to be the same for both diseases)
        \item The graph follows a random degree distribution. In one case, they specify Poisson distribution in this paper.
    \end{enumerate}
    \item The "cavity method" can be used to find the expected number of nodes infected by disease A
    \item For disease B, analysis is complicated by the fact that we must condition on infection with disease B; however, it can still be found with the "cavity method"
    \item These analyses also yield the probability that each disease becomes an epidemic
    \item The epidemic thresholds (The values at which the parameters of the model dictate the disease will transition from non-epidemic to epidemic) are also determined analytically
    \begin{enumerate}
        
        \item The epidemic threshold of the disease B decreases as the transmissibility of the disease A increases
        
        \item When the transmissibility of disease A is one (everything is infected disease A), then, disease B has the same epidemic threshold as disease B
        
    \end{enumerate}
    \item Two ways to stop the spread of disease B: limit transmissibility of A or B
    \item Perhaps, the transmission of the severe cases of COVID-19 could modeled under this framework of subsequent infection with disease A being one of the comorbidities with a high odds-ratio ( hypertension, respiratory system disease, cardiovascular
    disease)
    \item Based on the need for numerical iteration to solve for thresholds in the simple case of the Poisson Distribution, there may be potential for work on approximating these values in more complex situations. 
    
\end{enumerate}
\smallskip
\rule{\textwidth}{0.01in}

\textbf{Paper: } Stanoev et al., Modeling the spread of multiple contagions on networks.

\begin{enumerate}
    \item Model
    \item \begin{enumerate}
        \item General framework for an arbitrary number of contagions.
        \item Assumes all edges share the infection probability for each disease. I.E., there are global rates $\beta_{1},\beta_{2},...$
        \item Provides a numerical approximation for the update rule that closely fits the true update
        \item Handles multiple successful infections by choosing one successful infection uniformly randomly.
    \end{enumerate}
\end{enumerate}

\smallskip
\rule{\textwidth}{0.01in}

\textbf{Paper:} Caccioli et al., "Overlapping portfolios, contagion, and financial stability" 


\begin{enumerate}
    \item All nodes are a bank from a dataset of Austrian banks
    \item Edges represent the interbank loans
    \item Each bank has attributes for liquid and illiquid assets
    \item There is a single contagion of bankruptcy spread through multiple channels
    \begin{enumerate}
    \item The contagion of counterparty risk is spread by the risk on banks defaulting on their obligations to their lenders.
    \item The contagion of overlapping portfolios is represented by each node holding a common asset.
    \item Fire-sales of assets occur when a bank collapses and sells the common asset decreasing its price.
    \end{enumerate}
    \item The phenomena of fire-sales and common assets amplify the risk of the contagion spreading that is presented by interbank loans
    \item The primary finding of this paper is that none of the source of risks by itself is enough to create sufficient systemic risk without being of incredible proportions.
\end{enumerate}
\smallskip

\noindent
\rule{\textwidth}{0.01in}

\clearpage

\section{Influence Maximization}
\noindent

\textbf{Paper:}~
Chen et al., "Robust Influence Maximization", \emph{KDD '16: Proceedings of the 22nd ACM SIGKDD International Conference on Knowledge Discovery and Data Mining}, 2016, pgs. 795-804.
\begin{enumerate}
    \item Model: \begin{enumerate}
        \item The paper follows the single-contagion independent cascade model.
        \item Each infected node attempts to spread its contagion to susceptible neighbors each time step. Each edge infects with mutually independent probabilities.
    \end{enumerate}
    \item The greedy 1-$\frac{1}{e}$ heuristic of KKT (Kempe, Kleinberg, Tardos) is introduced for the independent cascade influence maximization problem. 
    \item The robust influence maximization problem is introduced in which edges have a confidence interval instead of a point estimate for the activation probability.
    \item \begin{enumerate}
        \item To deal with the robust problem, the KKT heuristic is used with the lower-bound of the confidence interval. This estimate is $\alpha(\theta)(1-\frac{1}{e}$ where $\alpha(\theta)$ = (spread from the KKT solution using the lower-bound)/(spread from the KKT solution using the upper-bound)
        \item To improve the quality of the solution, the edge probability distribution can be sampled according to a uniform and adaptive scheme. Refer to the paper for the full description of these.
    \end{enumerate}
    
\end{enumerate}
\noindent
\rule{\textwidth}{0.01in}
\clearpage
\textbf{Paper:}~ Wei Chen, Yajun Wang, and Siyu Yang. 2009. Efficient influence maximization in social networks. In Proceedings of the 15th ACM SIGKDD international conference on Knowledge discovery and data mining (KDD ’09). Association for Computing Machinery, New York, NY, USA, 199–208.
\begin{enumerate}
    \item Model: This paper proposes heuristics for the independent and weighted cascade models. I think only the independent cascade model is relevant to us.
    \item The new heuristic attempts to handle the large number of iterations that the Kempe,Kleinberg,Tardos (KKT) model incurs while estimating the influence spread through Monte Carlo simulations.
    \item Several new methods are proposed: \begin{enumerate}
        \item NewGreedyIC \begin{enumerate}
            \item Rather than run R Monte Carlo simulations for each node, the NewGreedyIC generates R subgraphs and computes the number of reachable nodes for each node in the graph on each particular subgraph.
            \item This provides roughly the same quality of estimation as the original KKT approximation with a lower cost.
        \end{enumerate}
        \item SingleDiscount \begin{enumerate}
            \item Uses the highest degree centrality measure as a heuristic to choose seed nodes.
            \item When a seed is chosen, it's neighbors degrees are decremented.
        \end{enumerate}
        \item DegreeDiscountIC \begin{enumerate}
            \item This heuristic uses a new degree discount heuristic that accounts for the probability of one of the node's selected neighbors influencing the node's neighbors through it.
            \item For the case of equal edge weights, DegreeDiscountIC tracks the performance of the original KKT algorithm.
        \end{enumerate}
    \end{enumerate}
\end{enumerate}


\noindent
\rule{\textwidth}{0.01in}
\clearpage
\textbf{Paper:}~ Wei Chen, Chi Wang, and Yajun Wang. 2010. Scalable influence maximization for prevalent viral marketing in large-scale social networks. In Proceedings of the 16th ACM SIGKDD international conference on Knowledge discovery and data mining (KDD ’10). Association for Computing Machinery, New York, NY, USA, 1029–1038. 
\begin{enumerate}
    \item This paper presents a $1 - \frac{1}{e}$ approximation algorithm for the single-contagion influence maximization problem.
    \item They present MIA (Maximum influence approximation) which greedily adds seeds by choosing those with the greatest marginal influence.
    \item The marginal influence of a potential seed node is estimated by removing areas of the network which the infection is unlikely to spread to from this node.
    
\end{enumerate}
  

\noindent
\rule{\textwidth}{0.01in}
\clearpage
\textbf{Paper:}~ 
S. Myers and J. Leskovec, ``Clash of the Contagions: Cooperation and
Competition in Information Diffusion", \emph{Proc. IEEE Intl. Conf. Data Mining}
(ICDM), 2012, pp. 539--548.

\begin{enumerate}
    \item Model 
    \begin{enumerate}
        \item Users sees a piece of content
        \item User decides whether or not to interact with a piece of content
        \item The sequence length and number of clusters is fixed
        \item The sequence of exposures is independent
        \item Contagions are not mutually exclusive
        \item Goal: Find the probability that a user interacts X with a piece of content given a sequence of K exposures
         \item How: Minimize the log-likelihood
        \item Three Mechanisms in This Decision: 
        \begin{enumerate}
            \item Virality- The inherent interestingness of the content
            \item User Bias- The user's preference for this type of content
            \item Content Interaction Term- How does what the user has seen impact the probability. This is treated as an additive term.
        \end{enumerate}
        \item Contagions are clustered for the purpose of the Content Interaction Term. This reduces the number of possible sequences.
        \item Interaction matrix fit with Stochastic Gradient Descent
        \item Idea: Multiply the virality term by an emotional valence term 
        \item Idea: Add a term that subtracts or adds to the probability based on the strength of an edge. This could be used to show how strong, dense sub-networks might be more likely to share a piece information originating within themselves although it has low virality.
    \end{enumerate}
    \item Dataset:
    \begin{enumerate}
        \item Data is sourced from Twitter
        \item The focus is on tweets the contain URLs
        \item Each URL is a contagion
        \item Following represents exposure to a contagion
        \item Re-tweeting represents infection with a contagion
    \end{enumerate}
    \item Highly infectious URLs increase the infectivity of URLs in their clusters and decrease the infectivity of those in other clusters
    
\end{enumerate}
\noindent
\rule{\textwidth}{0.01in}

\clearpage


\section{Co-existence Thresholds}

\noindent
\textbf{Paper:}~  
A. Beutel, B. A. Prakash, R. Rosenfeld and C. Faloutsos,
``Interacting Viruses in Networks: Can Both Survive?",
\emph{Proc. ACM KDD}, 2012, pp. 426--434.

\begin{enumerate}
    
    \item In a situation where two viruses compete but may co-exist, when can both survive as epidemics?
    
    \item Dataset: Google query counts of two partially competing products over time.
    
    \item Model \begin{enumerate}
        \item Assumes as $SI_{1}I_{2}S$ model
        
        \item Nodes return to S with rates $\delta_{1},\delta_{2}$ for infections of disease 1 and 2.
        
        \item Infections with diseases 1 and 2 spread at rates $\beta_{1}\beta_{2}$ respectively
        
        \item $\epsilon \in [0,1]$ damps the rate of infection if a node is already infected with the another disease. For example, if already infected with disease 1, a node is infected by disease 2 at rate $\epsilon\beta_{2}$.
        
        \item All nodes share these same parameters
        
        \item The underlying graph is a clique; this assumes 
        that there is homogeneous mixing. 
         
        
    \end{enumerate}
    
    \item The co-existences threshold is determined by functions of the ratios of infectivity to recovery rate for two diseases.
    
    
    \item The threshold for a fixed-point that co-exists can be 
    found analytically using differential equations describing the rate of infection within the clique.
    
    \item Primary contribution \begin{enumerate}
        \item There is a phase transition from co-existence to winner-takes-all
    \end{enumerate}
    
\end{enumerate}
\noindent
\rule{\textwidth}{0.01in}

\clearpage

\noindent
\textbf{Paper:}~ 
B. Karrer and M. E. J. Newman, ``Competing Epidemics on Complex Networks",
Arxiv Report arXiv:1105.3424v1, 2011, 14 pages.

\begin{enumerate}
    
    \item Model: \begin{enumerate}
        \item Cross-immunity is assumed for the diseases. If a node 
        recovers from disease 1, then, it cannot be infected by disease 2.
        
        \item Two diseases spread concurrently.
        
        \item Reed-Frost SIR model is adopted. The model uses discrete time steps and a uniform transmission probability for each disease.
        
        \item The network is directed. Infected seed nodes
        are chosen randomly from the network. 
        
        \item The two diseases spread in a fixed but different
        number of time-steps.
    \end{enumerate}
    
    \item Analysis in the limit of Large Networks
    \begin{enumerate}
        
        \item For large networks, analysis can produce a phase diagram, but for small networks, finite size effects limit this ability.
        
        \item Let $\beta$ be the ratio of the fast transmission rate for the fast-spread disease to the slow-spreading disease.
        
        \item For $\beta \neq 1$, the analysis of the network boils down to the residual network analysis of two diseases spreading subsequently. This is because the proportion of the network that may be infected by the less fit disease is always a fraction of the fit disease's infection.
        
        \item The spread of the slower disease can only occur if the fast disease has a sufficiently low transmission rate.
        
        \item If one disease starts with more carriers, 
        they may easily dominate the network.
    \end{enumerate}
    \item  
    
\end{enumerate}




\end{document}
