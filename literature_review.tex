\documentclass[11pt]{article}

\setlength{\textheight}{9.0 in}
\setlength{\textwidth}{6.5 in}
\setlength{\oddsidemargin}{0 in}
\setlength{\evensidemargin}{0 in}
\setlength{\topmargin}{-0.5 in}

\setlength{\parskip}{4pt}

%% \usepackage[nofillcomment,ruled,linesnumbered]{algorithm2e}
\usepackage{amsmath}
\usepackage{amssymb}
\usepackage{color}
\usepackage{graphicx}
\usepackage{url}

%% \pagestyle{empty}
\begin{document}

\newtheorem{theorem}{Theorem}[section]
\newtheorem{lemma}{Lemma}[section]
\newtheorem{corollary}{Corollary}[section]
\newtheorem{fact}{Fact}[section]
\newtheorem{definition}{Definition}[section]
\newtheorem{proposition}{Proposition}[section]
\newtheorem{observation}{Observation}[section]
\newtheorem{claim}{Claim}[section]

\newcommand{\cnp}{\textbf{NP}}
\newcommand{\true}{\texttt{True}}
\newcommand{\false}{\texttt{False}}

\newcommand{\QED}{\hfill\rule{2mm}{2mm}}

\newcommand{\irange}{\mbox{$1 \leq i \leq n$}}
\newcommand{\jrange}{\mbox{$1 \leq j \leq m$}}

\newcommand{\dunder}[1]{\underline{\underline{#1}}}

\setlength{\parskip}{3pt}

\baselineskip=1.1\normalbaselineskip

\begin{center}
\dunder{\Large{\textbf{Literature Review}}}
\end{center}

\medskip

\noindent
\textbf{Paper:}  P. Kumar, P. Verma, A. Singh and H. Cherifi,
``Choosing Optimal Seed Nodes in Competitive Contagion”, Frontiers
in Big Data, Vol. 2, Article 6, June 2019, pp. 1--6.

\medskip

\noindent
\textbf{Summary:}


\begin{itemize}
\item Focused on influence maximization
  \begin{itemize}
     \item two competing infections
     \item seed nodes chosen according to centrality measure; 
           higher centrality nodes have a high costs; 
           cost of the nodes is determined by best centrality measure
    \item less quantity, more influential nodes v. more quantity, less influential
        \begin{itemize}
	   \item used best centrality measure for given dataset
	   \item high-less is best across the board for all datasets
        \end{itemize}
	\item studies appear from a game theoretic setting; use of Nash equilibrium
	\item undirected network
	\item measures: degree centrality, closeness centrality
	(1/sum of distance from other nodes), betweenness centrality
	(percentage of shortest paths v appears in), the maximum
	eigenvalue of a node’s eigenvalue in the adjacency matrix,
	pagerank
	\item choices made in first experiment by order of decreasing centrality
	\item no one centrality measure is best
	\item equal competitive constants
	\item no conclusive evidence on using one centrality measure
            \begin{itemize}
		\item search algorithm for picking nodes not in
		order of descending centrality but based off expected
		benefit
            \end{itemize}
	\item cascade algorithm attempts to spread to all adjacent
	nodes; every time it is infected the competitive constant
	for the infecting strain grows on that node
   \end{itemize}
\end{itemize}

\smallskip
\rule{\textwidth}{0.01in}
\textbf{Paper:} S. Bharathi, D. Kempe, and M. Salek, "Competitive Influence Maximization in Social Networks" Dept. of CS, UCS

\begin{enumerate}
    \item Model:
    \begin{enumerate}
        \item Diffusion model
        
        \item Each edge has an activation probability
        
        \item The first activated incoming edge of a node determines the which infection it has
        
        \item The time at which the edge activation is determined by continuous independent, exponential r.v.'s
        
        \item Weakness: Once a node is activated, it has no chance of changing sets.$\implies$ Every graph reaches a fixed point
    \end{enumerate}
    \item Agent $i$ wants to maximize $E[|T_{i}|]$
    \item Strategy for the last player to commit
    \begin{enumerate}
    
        \item Greedy choice of adding node with highest 
        marginal return is within $1-\frac{1}{e}$ of optimal
        
        \item Uses fact that the nodes chosen according to this strategy add at least as many paths as those according to another strategy
        
    \end{enumerate}
    
    \item The Nash Equilibrium expected set of activated nodes is at least half that of if one person played
    
    \item First mover strategy in duopoly
    \begin{enumerate}
        \item Strongly polynomial dynamic program to maximize r's pieces
    \end{enumerate}
    
    \item FPTAS available for tree version of problem
    
    \item No greedy algorithm for more than duopoly and not last player
    
    \item Terms: 
    \begin{enumerate}
    
    \item Submodular- as the number of neighbors activates increases, their marginal impact decreases
    \\ i.e. for $S,T \subset V$, $f(S) + f(T) \geq f(S \cap T) + f(S \cup T)$
    
    \item Monotonicity- as the number of activated neighbors grows, the likelihood of activation increases
    \\ i.e. $f(S) \leq f(T) \leq f(V)$ \\ for $S \subset T \subset V$
    
    \end{enumerate}
    
\end{enumerate}
\smallskip
\noindent
\rule{\textwidth}{0.01in}
\textbf{Note from GDS Paper}
\begin{enumerate}
    \item Specified by $\mathcal{G, F, U}$
    \begin{enumerate}
        \item $\mathcal{F}$ uses inputs only from time $t$ to determine update at $t+1$
        \item $\mathcal{F}$ gives local transition functions
        \item a transition function $f$ is $r$ symmetric if it follows the threshold model i.e. $r=1,f(\Sigma a_{i}), r=2 f(\Sigma a_{i}, \Sigma a_{j}), ... etc.$ where if each of the sums is above a threshold $f$ is 1 and 0 otherwise.
    \end{enumerate}
    \item A Garden of Eden configuration has no predecessor
    \item A fixed-point is configuration that updates to itself
    
    \item Three analysis problems covered in depth:
    \begin{enumerate}
        \item Fixed-Point Existence
        \begin{enumerate}
            \item Checking is trivial
            \item NP-Complete in general but efficiently solvable in a lot of cases
        \end{enumerate}
        \item Reachability
        \begin{enumerate}
            \item Symmetric functions make this easier
            \item Intractable for heterogeneous symmetric functions
            \item Efficient for bi-threshold, acyclic, directed graphs
        \end{enumerate}
        \item Predecessor Existence
        \begin{enumerate}
            \item Existence + counting version are solvable for r-symmetric and treewidth bounded graphs
            \item In general, however, is NP-Complete
        \end{enumerate}
    \end{enumerate}
\end{enumerate}

\smallskip
\noindent
\rule{\textwidth}{0.01in}
\textbf{Paper:} M. Newman and C. Ferrario, "Interacting Contagion and Coinfection on Contact Networks." Plos One
\begin{enumerate}
    \item Model Assumptions:
    \begin{enumerate}
        \item Sequential Infections-Disease A first propogates through the network. Disease B then propogates after it.
        \item A node must first have been infected by disease A if it is to be infected by disease B
        \item Follows the SIR model
        \item Infected individuals become recovered at a fixed-time $\tau$ and infect neighbors with fixed probability $\beta$
        \item This model assumes the same mode of transmission (This is required for the network to be the same for both diseases)
        \item The graph follows a random distribution based on a Poisson random variable and the average degree
    \end{enumerate}
    \item The "cavity method" can be used to find the expected number of nodes infected by disease A
    \item For disease B, analysis is complicated by the fact that we must condition on infection with disease B; however, it can still be found with the "cavity method"
    \item These analyses also yield the probability that each disease becomes an epidemic
    \item The epidemic thresholds (The values at which the parameters of the model dictate the disease will transition from non-epidemic to epidemic) are also determined analytically
    \begin{enumerate}
        
        \item The epidemic threshold of the disease B decreases as the transmissibility of the disease A increases
        
        \item When the transmissibility of disease A is one (everything is infected disease A), then, disease B has the same epidemic threshold as disease B
        
    \end{enumerate}
    \item Two ways to stop the spread of disease B: limit transmissibility of A or B
    \item Perhaps, the transmission of the severe cases of Covid-19 could modelled under this framework of subsequent infection with disease A being one of the comorbidities with a high odds-ratio ( hypertension, respiratory system disease, cardiovascular
    disease)
    \item Based on the need for numerical iteration to solve for thresholds in the simple case of the Poisson Distribution, there may be potential for work on approximating these values in more complex situations. 
    
\end{enumerate}
\smallskip
\noindent
\rule{\textwidth}{0.01in}
\textbf{Paper:} S. Myers and J. Leskovec, "Clash of the Contagions: Cooperation and Competition in Information Diffusion", ICDM
\begin{enumerate}
    \item Model 
    \begin{enumerate}
        \item Users sees a piece of content
        \item User decides whether or not to interact with a piece of content
        \item The sequence length and number of clusters is fixed
        \item The sequence of exposures is independent
        \item Contagions are not mutually exclusive
        \item Goal: Find the probability that a user interacts X with a piece of content given a sequence of K exposures
         \item How: Minimize the log-likelihood
        \item Three Mechanisms in This Decision: 
        \begin{enumerate}
            \item Virality- The inherent interestingness of the content
            \item User Bias- The user's preference for this type of content
            \item Content Interaction Term- How does what the user has seen impact the probability. This is treated as an additive term.
        \end{enumerate}
        \item Contagions are clustered for the purpose of the Content Interaction Term. This reduces the number of possible sequences.
        \item Interaction matrix fit with Stochastic Gradient Descent
        \item Idea: Multiply the virality term by an emotional valence term 
        \item Idea: Add a term that subtracts or adds to the probability based on the strength of an edge. This could be used to show how strong, dense sub-networks might be more likely to share a piece information originating within themselves although it has low virality.
    \end{enumerate}
    \item Dataset:
    \begin{enumerate}
        \item Data is sourced from Twitter
        \item The focus is on tweets the contain URLs
        \item Each URL is a contagion
        \item Following represents exposure to a contagion
        \item Re-tweeting represents infection with a contagion
    \end{enumerate}
    \item Highly infectious URLs increase the infectivity of URLs in their clusters and decrease the infectivity of those in other clusters
    
\end{enumerate}



\end{document}
