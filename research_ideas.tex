\documentclass[11pt]{article}

\setlength{\textheight}{9.0 in}
\setlength{\textwidth}{6.5 in}
\setlength{\oddsidemargin}{0 in}
\setlength{\evensidemargin}{0 in}
\setlength{\topmargin}{-0.5 in}

\setlength{\parskip}{4pt}

%% \usepackage[nofillcomment,ruled,linesnumbered]{algorithm2e}
\usepackage{amsmath}
\usepackage{amssymb}
\usepackage{color}
\usepackage{graphicx}
\usepackage{url}

%% \pagestyle{empty}
\begin{document}

\newtheorem{theorem}{Theorem}[section]
\newtheorem{lemma}{Lemma}[section]
\newtheorem{corollary}{Corollary}[section]
\newtheorem{fact}{Fact}[section]
\newtheorem{definition}{Definition}[section]
\newtheorem{proposition}{Proposition}[section]
\newtheorem{observation}{Observation}[section]
\newtheorem{claim}{Claim}[section]

\newcommand{\cnp}{\textbf{NP}}
\newcommand{\true}{\texttt{True}}
\newcommand{\false}{\texttt{False}}

\newcommand{\QED}{\hfill\rule{2mm}{2mm}}

\newcommand{\irange}{\mbox{$1 \leq i \leq n$}}
\newcommand{\jrange}{\mbox{$1 \leq j \leq m$}}

\newcommand{\dunder}[1]{\underline{\underline{#1}}}

\setlength{\parskip}{3pt}

\baselineskip=1.1\normalbaselineskip

\newcommand{\cone}{\mbox{$\mathcal{C}_1$}}
\newcommand{\ctwo}{\mbox{$\mathcal{C}_2$}}


\newcommand{\sstate}{\mbox{$\mathbb{S}$}}
\newcommand{\istate}{\mbox{$\mathbb{I}$}}
\newcommand{\rstate}{\mbox{$\mathbb{R}$}}

\begin{center}
\dunder{\Large{\textbf{Ideas for Experiments}}}
\end{center}

\medskip

\section{Influence of Graph Structure on the Co-existence of Contagions} 

\subsection{Model}

We will use the SIR epidemic model with some modifications.
We consider two contagions denoted by \cone{} and \ctwo. The underlying
graph $G(V,E)$ is directed. Each node in $V$
can be in one of the following five possible states: 
\begin{description}
\item{(i)} \sstate{} (Susceptible), 
\item{(ii)} $\istate_{1}$ (Infected by \cone{} only),
\item{(iii)} $\istate_{2}$ (Infected by \ctwo{} only), 
\item{(iv)} $\istate_{1,2}$ (infected by both \cone{} and \ctwo) and 
\item{(v)} Recovered (\rstate). 
\end{description}
Each directed edge $(u,v)$  will have two probability values, denoted by $p_1$
and $p_2$; here, $p_1$ ($p_2$) denotes the probability of transmitting contagion
\cone{} (\ctwo) from $u$ to $v$.
For each contagion, the transition time from state
\istate{} to state \rstate{} will be specified by a random
variable. 
(Note: We may need to use two different random variables
for the two contagions. These variables must have integer values 
because of the SyDS model.) 

\subsection{Experimental Idea 1}

Choose two seed nodes that have a similar out-degree and run a
simulation to see if the infections for both nodes converge to 
fixed points. Each contagion will have it's edge probabilities
selected from a random variable that holds constant across simulations.
Run simulations where the graph is tweaked  to exhibit more or less
of a given type of structure. Use the outcomes of these trails to
determine a threshold at which both infections co-exist in the
limit.

\medskip

\noindent
\textbf{Properties to Test:}

 \begin{enumerate}
    \item Degree distribution
    \item Assortativity:
       \begin{enumerate}
            \item $k_{nn}$ distribution
            \item Assortativity coefficient
        \end{enumerate}
    \end{enumerate}

\end{document}
