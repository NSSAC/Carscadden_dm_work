\documentclass[11pt]{article}

\setlength{\textheight}{9.0 in}
\setlength{\textwidth}{6.5 in}
\setlength{\oddsidemargin}{0 in}
\setlength{\evensidemargin}{0 in}
\setlength{\topmargin}{-0.5 in}

\setlength{\parskip}{4pt}

%% \usepackage[nofillcomment,ruled,linesnumbered]{algorithm2e}
\usepackage{amsmath}
\usepackage{amssymb}
\usepackage{color}
\usepackage{graphicx}
\usepackage{url}
\usepackage{cite}


%% \pagestyle{empty}
\begin{document}

\newtheorem{theorem}{Theorem}[section]
\newtheorem{lemma}{Lemma}[section]
\newtheorem{corollary}{Corollary}[section]
\newtheorem{fact}{Fact}[section]
\newtheorem{definition}{Definition}[section]
\newtheorem{proposition}{Proposition}[section]
\newtheorem{observation}{Observation}[section]
\newtheorem{claim}{Claim}[section]

\newcommand{\cnp}{\textbf{NP}}
\newcommand{\true}{\texttt{True}}
\newcommand{\false}{\texttt{False}}

\newcommand{\QED}{\hfill\rule{2mm}{2mm}}

\newcommand{\irange}{\mbox{$1 \leq i \leq n$}}
\newcommand{\jrange}{\mbox{$1 \leq j \leq m$}}

\newcommand{\dunder}[1]{\underline{\underline{#1}}}

\setlength{\parskip}{3pt}

\baselineskip=1.1\normalbaselineskip

\newcommand{\cone}{\mbox{$\mathcal{C}_1$}}
\newcommand{\ctwo}{\mbox{$\mathcal{C}_2$}}


\newcommand{\sstate}{\mbox{$\mathbb{S}$}}
\newcommand{\istate}{\mbox{$\mathbb{I}$}}
\newcommand{\rstate}{\mbox{$\mathbb{R}$}}

\newcommand{\ione}{\mbox{$\mathbb{I}_1$}}
\newcommand{\itwo}{\mbox{$\mathbb{I}_2$}}
\newcommand{\ionetwo}{\mbox{$\mathbb{I}_{1,2}$}}

\begin{center}
\dunder{\Large{\textbf{Ideas for Experiments}}}
\end{center}


\medskip

\section{Previous Work:}
The work in \cite{Beutel_Interacting} mainly focuses on infectivity
ratios to understand the co-existence threshold; furthermore, there
is an assumption of homogeneous mixing. 
So, it is of interest to investigate how 
the conclusions in \cite{Beutel_Interacting} are affected
when other graph structures are considered.
The work in \cite{Karrer_2011}
assumes cross-immunity between two contagions whereas 
this work will assume independence between the contagions. 
Under our assumption, we will also investigate the role played 
by the graph structure. 
Additionally, many of the conclusions
of \cite{Karrer_2011} focus on the co-existence threshold as the
number of nodes grows to infinity. They discuss some of the consequences
finite-size effects but do not explore this fully.

\section{Influence of graph structure on co-existence threshold} 

\subsection{Model}
We will use the SIR SyDS model with some modifications. There
are two contagions denoted by \cone{} and \ctwo. The underlying graph,
$G(V,E)$, is a directed graph where each node in $V$ may be in one of five
possible states, namely \sstate{} (Susceptible), 
\ione{} (Infected by \cone{} only), 
\itwo{} (Infected by \ctwo{} only), 
\ionetwo{} (Infected by both \cone{} and \ctwo{}), and
\rstate{} (Recovered). 
Each edge is associated with two probability values, denoted by $p_1$ and
$p_2$ that correspond to the probability of transmitting contagions
\cone{} and \ctwo{} respectively. 

The transition time from \sstate{} to \rstate{} will be specified by a random
variable shared by the contagions. A similar specification is used for
the transition from  \rstate{} to \sstate.
We assume that contagions are independent of each other.

\subsection{Experimental Idea 1}

Choose two seed nodes that have a similar out-degree and run a
simulation to see if the infections for both nodes converge to 
fixed points. Each contagion will have its edge probabilities
selected from a random variable that holds constant across simulations.
Simulations should be run where edge probabilities are selected
from several cases: ecological neutrality case (\cite{Lipsitch}
discusses the importance of this) where the probabilities are i.i.d.,
the case where one distribution is slightly more infectious, and
the case where one distribution is vastly more infectious.

Run simulations where the graph is tweaked  to exhibit more or less
of a given type of structure. Use the outcomes of these trails to
determine a threshold at which both infections co-exist in the
limit.

\medskip

\noindent
\textbf{Properties to Test:}
    \begin{enumerate}
        \item Degree distribution
        \item Assortativity:
        \begin{enumerate}
            \item $k_{nn}$ distribution
            \item Assortativity coefficient
        \end{enumerate}
    \end{enumerate}

\subsection{Two-contagion DegreeDiscountIC}
In \cite{DegreeDiscount}, Chen et al. describe a degree-based heuristic, DegreeDiscountIC, that selects seeds for the influence maximization problem by calculating the expected number of nodes infected by the node in consideration given that it is not infected by another node. This simple heuristic works quite well on graphs with low infection probabilities. Furthermore, it runs fast and outperforms other simple heuristics.

% Given contagions \cone{}, \ctwo{}, let us define an extension of this heuristic. Let us denote $i_{\mathcal{C}_{i,j},v}$ as the interaction term for node $v$ and contagions $\mathcal{C}_{i}$ and $\mathcal{C}_{j}$. If node $v$ is not infected with $\mathcal{C}_{i}$ or $\mathcal{C}_{i}$ does interact with $\mathcal{C}_{j}$, $i_{\mathcal{C}_{i,j},v} = 0$. If the two contagions cooperate, $0 < i_{\mathcal{C}_{i,j},v}$. If the two contagions compete, then, $i_{\mathcal{C}_{i,j},v} < 0$. In the single contagion independent cascade model, a disease is spread between nodes with probability $p$. In the two contagion context, a disease $\cone{}$ is spread from node $u$ to $v$ with probability $\min\{1, p_{1}(1+i_{\mathcal{C}_{1,2},v})\}$; vice versa is true for the spread of $\ctwo{}$. Let $\mathcal{N}_{\cone{}}(u)$ denote the neighbors infected with \cone{} and $\mathcal{N}_(u)$ denote the neighbors of u. Similarly to the DegreeDiscountIC heuristic for contagion \cone{}, we will focus on the expected number of neighbors infected by some node $u$ in one time step given that $u$ is not infected by a neighbor with \cone{}. This is given by $(\prod_{v\in \mathcal{N}_{\cone{}z}(u)}(1-\min\{1, p_{1}(1+i_{\mathcal{C}_{2,1},u}\}))(1+\Sigma_{w \in  (\mathcal{N}(u) \mathcal{N}_{\cone{}z}(u))}(p_{\cone{}}(1-i_{\mathcal{C}_{2,1},w}))$. Let denote this quantity by $dd_{\cone{}, u}$. Note that the computation is analgous for \ctwo{}.
Before introducing a model for how two contagions interact, let us introduce a general form of the DegreeDiscountIC heuristic for multiple contagions. Let $\mathcal{C}_{i}(u)$ be the indicator random variable that denotes whether or not node $u$ is infected with $\mathcal{C}_{i}$. Let $\mathbf{I}_{C_{k}}(u,v)$ be the event that node $u$ transmits $\mathcal{C}_k$ to node $v$. The two contagion extension of DegreeDiscountIC for \cone{} is: $(\prod_{v\in \mathbf{}{N}_{\cone{}}(u)}(1-P[\mathbf{I}_{\cone}(v,u)|\ctwo(u) ]))(1+\Sigma_{w \in  (\mathbf{N}(u)-  \mathbf{N}_{\cone{}}(u))}(P[\mathbf{I}_{\cone}(u,w)|\ctwo(w)]))$, and the version for \ctwo{} is analogous.
    
Given contagions \cone{}, \ctwo{}, let us define an extension of this heuristic. Let us denote $i_{C_{k,j}}(v)$ as the additive interaction term for node $v$ and contagions $\mathcal{C}_{k}$ and $\mathcal{C}_{j}$. Let us restrict $i_{C_{k,j}}(u, v)$ to $[p_{\mathcal{C}_{k}}(u,v)+1, 1-p_{\mathcal{C}_{k}}(u,v)]$ where $p_{\mathcal{C}_{k}}(u,v)$ is the probability that node $u$ transmits $\mathcal{C}_{k}$ to node $v$. If node $v$ does not have $\mathcal{C}_j$, interaction term does nothing. If node $v$ has $\mathcal{C}_j$ and the two contagions cooperate, the transmission probability of $\mathcal{C}_k$ is now $p_{\mathcal{C}_{k}}(u,v) + i_{C_{k,j}}(v) > p_{\mathcal{C}_{k}}(u,v)$. If the contagions are independent, the transmission probability is unchanged. If they compete and node $v$ has $\mathcal{C}_j$, then,  the transmission probability is $p_{\mathcal{C}_{k}}(u,v) + i_{C_{k,j}}(v) < p_{\mathcal{C}_{k}}(u,v)$. We will call this model Additive Multiple Independent Cascade (AMIC). Similar to the regular independent cascade heuristic of Chen et al., we propose DegreeDiscountAMIC.   This is given by $(\prod_{v\in \mathbf{}{N}_{\cone{}}(u)}(1-p_{\cone{}}(v,u) + i_{C_{1,2}}(u)))(1+\Sigma_{w \in  (\mathbf{N}(u)-  \mathbf{N}_{\cone{}}(u))}(p_{\cone{}}(u,w)+ i_{C_{1,2}}(w)))$. Let denote this quantity by $dd_{\cone{}, u}$. Note that the computation is analgous for \ctwo{}.


Applying the degree heuristic to maximize the total number of infected nodes is not as simple as in the single contagion case of DegreeDiscountIC since choosing the seed node with the highest expected total infection due to contagion interaction. Some possible ideas that I have to deal with this are below:


\begin{enumerate}
    \item Naive: At every step, choose the node with the highest degree heuristic
    \item Initialize the seed set with the node with the contagion with the highest degree heuristic. For every subsequent, choose the node $v$ that maximizes $dd_{\mathcal{C}_i,v} + \Sigma_{u\in S}($increase in$ dd_{\mathcal{C}_i,u})$ where $S$ denotes the nodes currently in the seed set.
\end{enumerate}

For small graphs and seed sets, we can compare the new heuristic to an adaption of the KKT approximation algorithm to the multiple cooperating contagion case. This is described below:

$S = \emptyset$

for seed in range(k):

\indent for node in $G.V$

\indent\indent for $i$ in $\{\cone{}, \ctwo{}\}$

\indent\indent\indent for r in range(R)

\indent\indent\indent\indent $node.infected[i] += |RanCascade(S \bigcup \{node\})|$ 

\indent\indent\indent $node.infected[i] /= R$

\indent\indent $S = S \bigcup argmax_{v\in G.V}max_{i\in\cone{}, \ctwo{}} v.infected[i]$ 




\section{Financial Contagions}

In \cite{Cacciolo}, sources of risk that together create
source of systemic risk which allows an epidemic of bankruptcy to
spread through banking networks is discussed. There could be potential
to explore the interaction between financial contagion and other
forms such as social or disease.

\section{Graph Inference with Multiple Contagions}

There are many graph inference methods for social networks that
rely on a single social contagion. These could be extended for
networks on which we have information for multiple contagions to
improve the accuracy of inference.


%%%%%%%%%%%%%%%%%%  Bibliography %%%%%%%%%%%%%%%%%%

\bibliographystyle{plain}
\bibliography{references.bib}


\end{document}
