\documentclass[11pt]{article}

\setlength{\textheight}{9.0 in}
\setlength{\textwidth}{6.5 in}
\setlength{\oddsidemargin}{0 in}
\setlength{\evensidemargin}{0 in}
\setlength{\topmargin}{-0.5 in}

\setlength{\parskip}{4pt}

%% \usepackage[nofillcomment,ruled,linesnumbered]{algorithm2e}
\usepackage{amsmath}
\usepackage{amssymb}
\usepackage{color}
\usepackage{graphicx}
\usepackage{url}
\usepackage{cite}


%% \pagestyle{empty}
\begin{document}

\newtheorem{theorem}{Theorem}[section]
\newtheorem{lemma}{Lemma}[section]
\newtheorem{corollary}{Corollary}[section]
\newtheorem{fact}{Fact}[section]
\newtheorem{definition}{Definition}[section]
\newtheorem{proposition}{Proposition}[section]
\newtheorem{observation}{Observation}[section]
\newtheorem{claim}{Claim}[section]

\newcommand{\cnp}{\textbf{NP}}
\newcommand{\true}{\texttt{True}}
\newcommand{\false}{\texttt{False}}

\newcommand{\QED}{\hfill\rule{2mm}{2mm}}

\newcommand{\irange}{\mbox{$1 \leq i \leq n$}}
\newcommand{\jrange}{\mbox{$1 \leq j \leq m$}}

\newcommand{\dunder}[1]{\underline{\underline{#1}}}

\setlength{\parskip}{3pt}

\baselineskip=1.1\normalbaselineskip

\newcommand{\cone}{\mbox{$\mathcal{C}_1$}}
\newcommand{\ctwo}{\mbox{$\mathcal{C}_2$}}


\newcommand{\sstate}{\mbox{$\mathbb{S}$}}
\newcommand{\istate}{\mbox{$\mathbb{I}$}}
\newcommand{\rstate}{\mbox{$\mathbb{R}$}}

\newcommand{\ione}{\mbox{$\mathbb{I}_1$}}
\newcommand{\itwo}{\mbox{$\mathbb{I}_2$}}
\newcommand{\ionetwo}{\mbox{$\mathbb{I}_{1,2}$}}


\begin{center}
\dunder{\Large{\textbf{Ideas for Experiments}}}
\end{center}


\medskip

\section{Previous Work:}
The work in \cite{Beutel_Interacting} mainly focuses on infectivity
ratios to understand the co-existence threshold; furthermore, there
is an assumption of homogeneous mixing. 
So, it is of interest to investigate how 
the conclusions in \cite{Beutel_Interacting} are affected
when other graph structures are considered.
The work in \cite{Karrer_2011}
assumes cross-immunity between two contagions whereas 
this work will assume independence between the contagions. 
Under our assumption, we will also investigate the role played 
by the graph structure. 
Additionally, many of the conclusions
of \cite{Karrer_2011} focus on the co-existence threshold as the
number of nodes grows to infinity. They discuss some of the consequences
finite-size effects but do not explore this fully.

\section{Influence of graph structure on co-existence threshold} 

\subsection{Model}
We will use the SIR SyDS model with some modifications. There
are two contagions denoted by \cone{} and \ctwo. The underlying graph,
$G(V,E)$, is a directed graph where each node in $V$ may be in one of five
possible states, namely \sstate{} (Susceptible), 
\ione{} (Infected by \cone{} only), 
\itwo{} (Infected by \ctwo{} only), 
\ionetwo{} (Infected by both \cone{} and \ctwo{}), and
\rstate{} (Recovered). 
Each edge is associated with two probability values, denoted by $p_1$ and
$p_2$ that correspond to the probability of transmitting contagions
\cone{} and \ctwo{} respectively. 

The transition time from \sstate{} to \rstate{} will be specified by a random
variable shared by the contagions. A similar specification is used for
the transition from  \rstate{} to \sstate.
We assume that contagions are independent of each other.

\subsection{Experimental Idea 1}

Choose two seed nodes that have a similar out-degree and run a
simulation to see if the infections for both nodes converge to 
fixed points. Each contagion will have its edge probabilities
selected from a random variable that holds constant across simulations.
Simulations should be run where edge probabilities are selected
from several cases: ecological neutrality case (\cite{Lipsitch}
discusses the importance of this) where the probabilities are i.i.d.,
the case where one distribution is slightly more infectious, and
the case where one distribution is vastly more infectious.

Run simulations where the graph is tweaked  to exhibit more or less
of a given type of structure. Use the outcomes of these trails to
determine a threshold at which both infections co-exist in the
limit.

\medskip

\noindent
\textbf{Properties to Test:}
    \begin{enumerate}
        \item Degree distribution
        \item Assortativity:
        \begin{enumerate}
            \item $k_{nn}$ distribution
            \item Assortativity coefficient
        \end{enumerate}
    \end{enumerate}


\section{Financial Contagions}

In \cite{Cacciolo}, sources of risk that together create
source of systemic risk which allows an epidemic of bankruptcy to
spread through banking networks is discussed. There could be potential
to explore the interaction between financial contagion and other
forms such as social or disease.

\section{Graph Inference with Multiple Contagions}

There are many graph inference methods for social networks that
rely on a single social contagion. These could be extended for
networks on which we have information for multiple contagions to
improve the accuracy of inference.


%%%%%%%%%%%%%%%%%%  Bibliography %%%%%%%%%%%%%%%%%%

\bibliographystyle{plain}
\bibliography{references.bib}


\end{document}
