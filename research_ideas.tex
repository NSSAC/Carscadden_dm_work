\documentclass[11pt]{article}

\setlength{\textheight}{9.0 in}
\setlength{\textwidth}{6.5 in}
\setlength{\oddsidemargin}{0 in}
\setlength{\evensidemargin}{0 in}
\setlength{\topmargin}{-0.5 in}

\setlength{\parskip}{4pt}

%% \usepackage[nofillcomment,ruled,linesnumbered]{algorithm2e}
\usepackage{amsmath}
\usepackage{amssymb}
\usepackage{color}
\usepackage{graphicx}
\usepackage{url}

%% \pagestyle{empty}
\begin{document}

\newtheorem{theorem}{Theorem}[section]
\newtheorem{lemma}{Lemma}[section]
\newtheorem{corollary}{Corollary}[section]
\newtheorem{fact}{Fact}[section]
\newtheorem{definition}{Definition}[section]
\newtheorem{proposition}{Proposition}[section]
\newtheorem{observation}{Observation}[section]
\newtheorem{claim}{Claim}[section]

\newcommand{\cnp}{\textbf{NP}}
\newcommand{\true}{\texttt{True}}
\newcommand{\false}{\texttt{False}}

\newcommand{\QED}{\hfill\rule{2mm}{2mm}}

\newcommand{\irange}{\mbox{$1 \leq i \leq n$}}
\newcommand{\jrange}{\mbox{$1 \leq j \leq m$}}

\newcommand{\dunder}[1]{\underline{\underline{#1}}}

\setlength{\parskip}{3pt}

\baselineskip=1.1\normalbaselineskip

\begin{center}
\dunder{\Large{\textbf{Experiment Ideas}}}
\end{center}

\medskip
\section{Influence of graph structures on co-existence} 
\subsection{Model}
    We will be using the SIR SyDS model with some modifications. There will be two
    contagions $I_{1}$ and $I_{2}$. The underlying graph, $G(V,E)$, will be a directed graph
    containing nodes $V$ with five possible states susceptible, $I_{1}, I{2}, I_{1}I_{2}$, and recovered. Each edge will contain 
    two probabilities $p_{I_{1}}$ and $p_{I_{2}}$ that correspond to the probability of transmitting $I_{1}$ and $I_{2}$ respectively. 
    The transition time from susceptible to recovered/removed will be specified by a random variable shared by the contagions and, likewise, for recovered/removed to susceptible. 
\subsection{Experiment Idea 1}
    Choose two seed nodes that have a similar out-degree and run a simulation to see if the infections for both nodes
    converge to a fixed points. Each contagion will have it's edge probabilities selected from a random variable that holds constant
    across simulations. Run simulations where the graph is tweaked  to exhibit more or less of a given type of structure. Use the outcomes of these trails to determine a threshold at which both infections co-exist in the limit. \\
    Properties to Test:
    \begin{enumerate}
        \item Degree distribution
        \item Assortativity:
        \begin{enumerate}
            \item $k_{nn}$ distribution
            \item assortativity coefficient
        \end{enumerate}
    \end{enumerate}
    


\end{document}
