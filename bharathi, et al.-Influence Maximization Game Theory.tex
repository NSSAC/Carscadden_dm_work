\documentclass{article}
\usepackage[utf8]{inputenc}

\title{bharathi, et al.-Influence Maximization Game Theory}
\author{hlc5v }
\date{April 2020}

\begin{document}

\maketitle

\begin{enumerate}

    \item Agent $i$ wants to maximize $E[|T_{i}|]$
    \item Strategy for the last player to commit
    \begin{enumerate}
    
        \item Greedy choice of adding node with highest 
        marginal return is within $1-\frac{1}{e}$ of optimal
        
        \item Uses fact that the nodes chosen according to this strategy add at least as many paths as those according to another strategy
        
    \end{enumerate}
    
    \item The Nash Equilibrium expected set of activated nodes is at least half that of if one person played
    
    \item First mover strategy in duopoly
    \begin{enumerate}
        \item Strongly polynomial dynamic program to maximize r's pieces
    \end{enumerate}
    
    \item FPTAS available for tree version of problem
    
    \item No greedy algorithm for more than duopoly and not last player
    
    \item Terms: 
    \begin{enumerate}
    
    \item Submodular- as the number of neighbors activates increases, their marginal impact decreases
    \\ i.e. for $S,T \subset V$, $f(S) + f(T) \geq f(S \cap T) + f(S \cup T)$
    
    \item Monotonicity- as the number of activated neighbors grows, the likelihood of activation increases
    \\ i.e. $f(S) \leq f(T) \leq f(V)$ \\ for $S \subset T \subset V$
    
    \end{enumerate}
    
\end{enumerate}

\end{document}
